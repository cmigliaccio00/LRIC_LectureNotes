\chapter{Lab 1: Least-Squares parameter estimation (Solution)}

In this problem we assume that the plant is a continuous time LTI dynamical system assumed
to be exactly described by the following transfer function:

\begin{equation*}
    G_p(s) = \frac{100}{s^2+1.2s+1}
\end{equation*}

\begin{lstlisting}[style=Matlab-editor]  
    G=100/(s^2+1.2*s+1);
\end{lstlisting}

\noindent
Assuming that the input-output data have been collected with a sample time of Ts = 1s,
compute a discrete-time model for the plant as follows (see help of the Matlab command
\texttt{c2d}): $G_d(z)=\textsf{c2d}(G_p,1,\text{'zoh'})$:

\begin{lstlisting}[style=Matlab-editor]  
    Gz=c2d(G,1,'zoh'); 
\end{lstlisting}

\noindent
Use the obtained discrete-time model to generate input-output data by applying to the system
a randomsequence of $N$ samples and amplitude 1 as input (see commands rand and command
\texttt{lsim}). Collect the input sequence in the array $u$ and the output sequence in the array $w$.

\begin{lstlisting}[style=Matlab-editor]    
    n=2; 
    k=n+1;              %Point where I have to start
    H=1e5+n;            %I need at least (3n+1) samples
            
    u=rand(H, 1);
    %noisy-data (Equation Error)
    w_nf=lsim(Gz,u);
\end{lstlisting}

\section{Noise-Free experiment}
\noindent
Using the array u and w, build matrix A and array b required for the computation of the least
squares estimate of the discrete-time model parameters, according to the theory and examples
about least square estimation presented in the classroom.

\begin{lstlisting}[style=Matlab-editor] 
    [numGz, denGz]=tfdata(Gz,'v');
    th_true = [denGz(2:end) numGz]'

    A = [-w_nf(2:H-1) -w_nf(1:(H-2)) u(3:H) u(2:(H-1)) u(1:(H-2))];
    th_est_noise_free = A\w_nf(k:H)
\end{lstlisting}

\section{Equation-Error setting}
Repeat the exercise (for different values of $N$) by adding a random \textit{equation error} $e$ while
simulating the data.

\begin{lstlisting}[style=Matlab-editor] 
    D=tf([0 0 1],denGz,1);
    e = 5*randn(H,1);
    w_nEE = lsim(Gz,u)+lsim(D,e);
    A = [-w_nEE(2:H-1) -w_nEE(1:(H-2)) u(3:H) u(2:(H-1)) u(1:(H-2))];
    th_est_noiseEE = A\w_nEE(k:H) 
\end{lstlisting}

\section{Output-Error setting}
Repeat the exercise (for different values of $N$) by adding a random \textit{output measurement error}
$\eta$.

\begin{lstlisting}[style=Matlab-editor] 
    %simulation in case of Output-Error(OE) setting
    std=5;
    eta=std*randn(H,1);
    w_nOE=lsim(Gz,u)+eta;

    A = [-w_nOE(2:H-1) -w_nOE(1:(H-2)) u(3:H) u(2:(H-1)) u(1:(H-2))];
    th_est_noiseOE = A\w_nOE(k:H)
\end{lstlisting}



