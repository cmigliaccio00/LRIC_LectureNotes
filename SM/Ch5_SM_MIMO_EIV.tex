\chapter[SM SysId of MIMO LTI systems with EIV noise structure]{Set-Membership Identification of MIMO LTI systems with EIV noise structure}

\begin{quotation}
    \textsf{\noindent Till now we have introduced fundamental aspects about Set-Membership Identification, we have understood why it is the most realistic approach to the SysId, and -- in order to go step by step -- we focused our attention on a specific case on which we developed the theory: \textit{SISO LTI systems with Errors-in-variables noise structure.} Our objective now is trying to generalize such results, including the case in which the system to identify is multi-input multi-output (MIMO) or nonlinear. In this chapter we treat the former topic.
    }
\end{quotation}

\noindent
Let us start introducing some general concepts and definition useful for the comprehension of the incoming topics. A \textbf{Multi-Input Multi-Output (MIMO)} Linear Time-Invariant system with $p$ inputs and $q$ outputs can be described by means of a \textit{matrix transfer function} $G(q^{-1})$ where each element represents a SISO transfer function 
\begin{equation}
    G_{ij}(q^{-1})=\frac{N_{ij}(q^{-1})}{D_{ij}(q^{-1})}
\end{equation} 
between the $j$-th input and the $i$-th output.\\
The I/O relationship of the system we want to study is defined by the following equation:
\begin{equation} \label{eq:MIMO_IO}
    \begin{bmatrix}
        y_1\\y_2\\\vdots\\y_q 
    \end{bmatrix}=\underbrace{\begin{bmatrix}
        G_{11}(q^{-1})&G_{12}(q^{-1})&\dots&G_{1p}{(q^{-1})}\\
        G_{21}(q^{-1})&G_{22}(q^{-1})&\dots&G_{2p}(q^{-1})\\
        \vdots&\vdots&\ddots&\vdots\\
        G_{q1}(q^{-1})&G_{q2}(q^{-1})&\dots&G_{qp}(q^{-1})
    \end{bmatrix}}_{G(q^{-1})}\begin{bmatrix}
        u_1\\u_2\\\vdots\\u_p
    \end{bmatrix}
\end{equation}
A block diagram representation of a MIMO system is reported below: 

\begin{figure}[h]
    \centering
    \includegraphics[scale=0.15]{img/MIMO.jpg}
    \caption{Block diagram for a MIMO system}
\end{figure}

\noindent
The reason why I use directly a transfer function description for the system I want to identify is the same for which we use it in the simple case of a SISO LTI system: since we are able to do some open-loop experiments on the plant collecting I/O data, an I/O description such as the \textit{regression form} (that leads to a transfer function) is the most convenient way to mathematically describe the system itself!\\
However, we can have an objection in the sense that we can find an (apparently) \textbf{useful insight} if we start from a state-space description, roughly speaking using the matrices $A,B,C,D$. Indeed, recalling what is the definition of (matrix) transfer function $H(s)$ obtained starting from the state space description we know that:
\begin{equation*}
    \underbrace{H(s)=C(sI-A)^{-1}{B}+D}_{\textsf{continuous time}}, \quad
    \underbrace{H(z)=C(zI-A)^{-1}B+D}_{\textsf{discrete time}} 
\end{equation*}
(From now on without loss of generality, for sake of simplicity we will use $(sI-A)$ for the explanation of what follows).
By computing the inverse of the matrix $sI-A$ we have to divide it by its determinant $\det(sI-A)$ (which is also the \textit{characteristic polynomial}). For this reason all of the elements of $H(s)$ have the same common denominator, resulting into the same parameters to be estimated! \\
Now, at the end of the day our problem is \textit{estimating some parameters} exploiting I/O experimental data. The approaches I can use in order to continue developing the theory are:
\begin{enumerate}
    \itemsep-0.3em
    \item Considering the denominators of the transfer functions \textbf{identical} how the state-space insight suggests us (this plays the role of an additive a-priori information); 
    \item Considering the transfer functions as having \textbf{different denominators} resulting in a greater number of parameters to be estimated.
\end{enumerate}

\noindent
It would be better to analyze the features for one approach and for the other. We can say that the most evident advantage in considering the same all of the denominators is that we have \textbf{less parameters} entering the identification procedure. On the other hand, in the second case we could have \textbf{some physical a-priori information} that suggest us something about the order of each transfer function, the order $n$ in general can be different from one transfer function to another. In such a case we know something directly related to the number of parameters to be estimated\footnote{
    For a system of order $n$, I have $2n+1$ parameter to estimate.
}. Apparently, we are anyway tempted to say that such an information is not lost in the first case, since there could be \textbf{zero-pole cancellations} which makes also very different the several transfer functions. However this is not true, since our collected data are affected by noise and then the parameters related to the zeros/poles are not exact. It is sufficient to think about the fact we retrieve some PUIs from the Set-Membership procedure, uncertainty is embedded into the problem. This evidence suggests us that maybe the best path to be followed is not the one that apparently makes the identification problem simpler. \textit{We will going on discussing the problem following the second approach.}\\

\noindent
Starting from the (\ref{eq:MIMO_IO}), a generic output of the system $y_i$ is given by:
\begin{equation}
    y_i(k)=G_{i1}(q^{-1})u_1(k)+G_{i2}(q^{-1})u_2(k)+\dots+G_{ip}(q^{-1})u_p(k), \quad i=1,...,q
\end{equation}
From this follows an \textbf{important remark}: each output $y_i$ depends on the past samples of the output itself and the samples of the inputs $u_1,...,u_p$ not by the other outputs. In other words the $y_i$ behaviours are \textbf{independent}. Therefore, a first conclusion can be drawn: \textit{the identifiction of a MIMO LTI system with $q$ outputs is equivalent to the identification of $q$ \textbf{MISO} (Multi-input single-output) systems}. Now, the sub-problem to be faced is:

\section{SM Identification of MISO LTI systems}
As usual in a set-membership identification procedure we have to list all of the ingredients of our problem and then put them together to obtain the feasible parameter set. This is the what we are going to do.

\subsection{A-priori information on the system}
The system order $n$ is assumed to be known moreover the I/O mapping can be expressed as follows: 
\begin{equation}
    y=G(q^{-1})\begin{bmatrix}
        u_1\\\vdots\\u_p
    \end{bmatrix} = \begin{bmatrix}
        G_1(q^{-1})&G_2(q^{-1})& \dots & G_p(q^{-1})
    \end{bmatrix}\begin{bmatrix}
        u_1\\\vdots\\u_p
    \end{bmatrix}
\end{equation}
where the function $G_i(q^{-1})$ is:
\begin{equation}\label{eq:Gi}
    G_i(q^{-1}) = \frac{
        \beta_0^i+\beta_1^i{q^{-1}}+...+\beta_{n_i}^i{q^{-n_i}}
    }{1+\alpha_1^i{q^{-1}}+...+\alpha_{n_i}^i{q^{-n_i}}
    }, \quad i=1,...,p
\end{equation}
$n_i\le{n}$ is the dynamical order of $G_i(q^{-1})$. The fact that $n_i$ could be smaller is derived from some other a-priori information, otherwise we put always $n_i=n$ for that function we do not know other insights.

\subsection{A-priori information on the noise}
With the objective of being as more general as possible, we analyze the case \textbf{errors-in-variables} (EIV) where both the inputs and the output are affected by measurement noise. A block diagram showing this set-up is given here:
\begin{figure}[h]
    \centering
    \includegraphics[scale=0.19]{img/MISO.jpg}
    \caption{Block diagram of a MISO LTI system with EIV noise structure}
\end{figure}

%TODO: figura MISO system with EIV noise structure
As in the SISO case we make some quite realistic assumptions on the boundedness of the noise samples, that is:
\begin{equation}
    \begin{aligned}
        &\vert \eta(k) \vert \le \Delta_\eta, \quad k=1,...,N\\
        &\vert \xi_i(k) \vert \le \Delta_{\xi_i} \quad k=1,...,N \quad i=1,...,p
    \end{aligned}
\end{equation}
where $N$ as usual is the number of I/O collected data, and $\Delta_{\eta}, \ \Delta_{\xi_i}$ are the only available information on the noise.

\subsection{A-posteriori information: experimental data}
In order to perform the identification $N$ samples of the inputs $\tilde{u}_1(k), ..., \tilde{u}_p(k), \quad k=1,...,N$ and the output $\tilde{y}(k)$ must be collected by doing an \textit{open-loop experiment} on the plant to be identified.

\subsection{Feasible Parameter set $\mathcal{D}_\theta$}
The next step is to put everything together in order to define the \textbf{feasible parameter set} $\mathcal{D}_\theta$:
\begin{equation}
    \begin{aligned}
        \mathcal{D}_\theta = \big\{
            \theta\in\mathbb{R}^{\sum_{i=1}^p{2n_i+1}}: \quad  
            &y(k)=G_1(q^{-1})u_1(k)+...+G_p(q^{-1})u_p(k),  \quad k=2n+1,...,N\\
            &y(k)=\tilde{y}(k)-\eta(k), \quad
            u_i(k)=\tilde{u}_i(k)-\xi_i(k), \quad i=1,...,p\\
            &
            \vert \eta(k) \vert \le \Delta_\eta, \quad \vert \xi_i(k) \vert \le \Delta_{\xi_i} \quad k=1,...,N \quad i=1,...,p
        \big\}
    \end{aligned}
\end{equation}
More explicitly we can substitute the transfer functions $G_i(q^{-1})$ with the definition we gave in (\ref{eq:Gi}), the set becomes:
\begin{equation}
    \begin{aligned}\label{eq:FPS_MIMO2}
        \mathcal{D}_\theta = \big\{
            &\theta\in\mathbb{R}^{\sum_{i=1}^p{2n_i+1}}: \quad
            y(k)={
                \frac{
                    \beta_0^1+\beta_1^1{q^{-1}}+...+\beta_{n_1}^1{q^{-n_1}}
                }{1+\alpha_1^1{q^{-1}}+...+\alpha_{n_1}^1{q^{-n_1}}
                }
            }u_1(k)+\\
            &+\dots
            +{
                \frac{
                    \beta_0^p+\beta_1^p{q^{-1}}+...+\beta_{n_p}^p{q^{-n_p}}
                }{1+\alpha_1^p{q^{-1}}+...+\alpha_{n_p}^p{q^{-n_p}}
                }
            }u_p(k),  \quad k=2n+1,...,N\\
            &y(k)=\tilde{y}(k)-\eta(k), \quad
            u_i(k)=\tilde{u}_i(k)-\xi_i(k), \quad i=1,...,p\\
            &
            \vert \eta(k) \vert \le \Delta_\eta, \quad \vert \xi_i(k) \vert \le \Delta_{\xi_i} \quad k=1,...,N \quad i=1,...,p
        \big\}
    \end{aligned}
\end{equation} 
Starting from this point, how can we going on? Also in this case we could be tempted in modeling the MISO as a set of SISO, but there are some drawbacks: 
\begin{itemize}
    \item Keep in mind that for a system of order $n$ we have to solve $2(2n+1)$ optimization problems for finding the parameter uncertainty intervals, then this will result in a high computation load; 
    \item The temptation raises up from the moment that we grasp to the \textit{superposition principle for LTI systems} according which if we apply the inputs one at a time, the output will show a behaviour which is related only to that input itself. Unfortunately, in real-world applications is not always possible to "turn-off" some inputs while keeping unchanged the system, this is also because the great majority of the plants are \textit{only approximatively linear}.
\end{itemize}

Another attempt could be doing the common denominator in  (\ref{eq:FPS_MIMO2}) and solving a unique huge identification problem. Again, still there are problems! The dimension of the problem quckly would explode making unsuitable \texttt{SparsePOP} for solving the problem exploiting convex relaxation techniques. Something different is needed... 
